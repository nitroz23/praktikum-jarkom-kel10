\section{Pendahuluan}
\subsection{Latar Belakang}
Jaringan Pribadi Virtual (VPN) adalah teknologi yang memungkinkan komunikasi aman melalui jaringan publik dengan menciptakan terowongan terenkripsi antara perangkat. Teknologi ini banyak digunakan oleh organisasi untuk menghubungkan kantor cabang atau karyawan ke jaringan pusat secara aman, memastikan bahwa data sensitif tetap terlindungi dari akses yang tidak sah. VPN beroperasi menggunakan berbagai protokol seperti IPSec, SSL/TLS, dan L2TP, masing-masing menawarkan tingkat enkripsi dan otentikasi yang berbeda. Dalam konteks bisnis, VPN sangat penting untuk menjaga kerahasiaan dan integritas transmisi data, terutama saat mengakses sumber daya melalui internet atau menghubungkan beberapa lokasi dalam jaringan area luas (WAN).

Quality of Service (QoS), di sisi lain, merujuk pada sekelompok teknologi yang digunakan untuk mengelola lalu lintas jaringan dengan memprioritaskan jenis data tertentu. Hal ini memastikan bahwa layanan kritis seperti suara, video, dan aplikasi real-time mendapatkan bandwidth yang diperlukan dan latensi rendah untuk kinerja optimal. Mekanisme QoS dapat mengalokasikan bandwidth, mengurangi kehilangan paket, dan mengelola kemacetan, yang sangat penting dalam lingkungan di mana aplikasi dan pengguna yang berbeda berbagi koneksi internet yang sama. Misalnya, di lembaga pendidikan atau perusahaan, QoS dapat memprioritaskan platform e-learning atau aplikasi bisnis kritis di atas lalu lintas yang kurang penting, seperti unduhan file atau akses media sosial.

Bersama-sama, VPN dan QoS memainkan peran krusial dalam manajemen jaringan modern—VPN menyediakan konektivitas aman, sementara QoS memastikan kinerja jaringan yang efisien dan andal. Implementasi kedua teknologi ini memungkinkan organisasi untuk mendukung akses jarak jauh, melindungi data sensitif, dan mempertahankan pengiriman layanan berkualitas tinggi di lingkungan jaringan terdistribusi.

\subsection{Dasar Teori}
Konsep Jaringan Pribadi Virtual (VPN) secara teoritis berakar pada disiplin ilmu keamanan jaringan, kriptografi, dan komunikasi data yang aman. VPN dirancang untuk meniru keamanan dan privasi jaringan pribadi melalui infrastruktur publik seperti internet. Pada dasarnya, VPN menggunakan protokol tunneling seperti IPSec (Internet Protocol Security), SSL/TLS (Secure Sockets Layer/Transport Layer Security), dan L2TP (Layer 2 Tunneling Protocol) untuk mengenkapsulasi paket data, sehingga menyembunyikan isi dan asal-usulnya. Enkapsulasi ini biasanya disertai dengan algoritma enkripsi yang kuat seperti AES (Advanced Encryption Standard) atau 3DES (Triple Data Encryption Standard) untuk memastikan kerahasiaan, serta fungsi hashing seperti SHA (Secure Hash Algorithm) untuk memastikan integritas data. Metode autentikasi seperti kunci pra-bagi (PSK) atau sertifikat digital digunakan untuk memverifikasi identitas pihak yang berkomunikasi, melindungi dari akses tidak sah dan serangan man-in-the-middle. VPN sangat bergantung pada prinsip-prinsip CIA triad—Kerahasiaan, Integritas, dan Ketersediaan—sebagai landasan teoretis. Dalam hal lapisan jaringan, VPN umumnya beroperasi di lapisan jaringan (Layer 3) atau lapisan transportasi (Layer 4) model OSI, memungkinkan pengiriman paket yang aman melalui jaringan yang beragam dan potensial tidak tepercaya.

Di sisi lain, kerangka kerja Quality of Service (QoS) didasarkan pada teori prioritas lalu lintas jaringan, manajemen bandwidth, dan komunikasi yang sensitif terhadap penundaan. Dalam jaringan, tidak semua paket data memiliki tingkat penting yang sama; misalnya, paket suara atau video real-time jauh lebih sensitif terhadap penundaan dan jitter daripada email atau unduhan file. QoS secara teoritis mengatasi hal ini dengan menggunakan klasifikasi dan penandaan lalu lintas untuk mengidentifikasi jenis-jenis lalu lintas jaringan yang berbeda, yang kemudian dikenakan tingkat layanan yang berbeda. Proses ini diimplementasikan melalui model seperti Integrated Services (IntServ), yang menggunakan sinyal eksplisit dan reservasi sumber daya untuk aliran individu, dan Differentiated Services (DiffServ), yang menerapkan penandaan lalu lintas (misalnya DSCP) dan perilaku per-hop tanpa memerlukan status per-aliran. Setelah lalu lintas diklasifikasikan, disiplin antrian seperti Priority Queuing, Weighted Fair Queuing (WFQ), dan Hierarchical Token Bucket (HTB) digunakan untuk menentukan urutan dan laju transmisi paket. Selain itu, mekanisme seperti pengendalian kemacetan, pembentukan lalu lintas, dan pembatasan laju diterapkan untuk memastikan sumber daya jaringan dialokasikan sesuai kebijakan dan permintaan.

Integrasi VPN dan QoS secara teoritis sangat penting dalam jaringan modern, terutama di lingkungan perusahaan dan institusi. Sementara VPN memastikan data tetap aman dan terlindungi dari penyadapan dan manipulasi, QoS memastikan data tersebut—terutama lalu lintas real-time dan kritis—dikirimkan secara efisien dengan penundaan, kehilangan, dan jitter minimal. Tanpa QoS, lalu lintas VPN yang terenkripsi mungkin mengalami kinerja buruk akibat kemacetan jaringan, sementara tanpa VPN, lalu lintas sensitif mungkin rentan terhadap penyadapan. Oleh karena itu, arsitektur jaringan yang secara teoritis kokoh menggabungkan VPN untuk jaminan keamanan dan QoS untuk jaminan kinerja, memastikan integritas dan kualitas layanan bagi pengguna di jaringan terdistribusi dan berbagi. 

%===========================================================%
\section{Tugas Pendahuluan}
\begin{enumerate}
	\item \textbf{Fase Negosiasi IPSec}
	IPSec menggunakan protokol IKE (Internet Key Exchange) dalam dua fase:

	\textbf{IKE Phase 1}
	Tujuannya adalah membangun ISAKMP Security Association (SA), yaitu kanal aman pertama.
	\begin{itemize}
		\item \textbf{Mode}: Main Mode (lebih aman) atau Aggressive Mode (lebih cepat, kurang aman)
		\item \textbf{Proses}:
		\begin{enumerate}
			\item Tukar parameter keamanan (encryption, hash, DH group)
			\item Otentikasi peer (pre-shared key, sertifikat)
			\item Negosiasi algoritma dan membuat SA
		\end{enumerate}
		\item \textbf{Output}: Terbentuk ISAKMP SA (Phase 1 SA) → lalu digunakan untuk melindungi Phase 2
	\end{itemize}

	\textbf{IKE Phase 2}
	Digunakan untuk membangun IPSec SA dan menyepakati parameter untuk lalu lintas data.
	\begin{itemize}
		\item \textbf{Mode}: Quick Mode (lebih cepat, fokus pada data)
		\item \textbf{Proses}:
		\begin{enumerate}
			\item Negosiasi protokol IPSec (ESP atau AH)
			\item Negosiasi algoritma enkripsi dan autentikasi
			\item Tukar keying material
			\item Membuat IPSec SA
		\end{enumerate}
		\item \textbf{Output}: Dua IPSec SA (untuk dua arah komunikasi)
	\end{itemize}
	\begin{center}
		\begin{tabular}{ |p{5cm}||p{7cm}| }
			\hline
			\multicolumn{2}{|c|}{Parameter Keamanan yang Harus Disepakati} \\
			\hline
			PARAMETER & KETERANGAN \\
			\hline
			Encryption Algorithm   & AES-256, 3DES, AES-128 \\
			Authentication Method & Pre-Shared Key (PSK), RSA Certificates \\
			Integrity Algorithm & SHA-256, SHA-1 \\
			DH Group & DH Group 14 (2048-bit), DH Group 5 (1536-bit) \\
			Lifetime & Phase 1: 86400 detik (24 jam), Phase 2: 3600 detik (1 jam) \\
			\hline
		\end{tabular}
	\end{center}

	\textbf{Contoh Konfigurasi Router}
	\begin{lstlisting}
		crypto isakmp policy 10
		encryption aes 256
		hash sha256
		authentication pre-share
		group 14
		lifetime 86400

		crypto isakmp key MYSECRETKEY address 203.0.113.2

		crypto ipsec transform-set MYSET esp-aes 256 esp-sha-hmac

		crypto map MYMAP 10 ipsec-isakmp
		set peer 203.0.113.2
		set transform-set MYSET
		match address 101

		access-list 101 permit ip 192.168.1.0 0.0.0.255 192.168.2.0 0.0.0.255

		interface GigabitEthernet0/1
		ip address 203.0.113.1 255.255.255.0
		crypto map MYMAP
	\end{lstlisting}
	\begin{itemize}
		\item 192.168.1.0/24: jaringan kantor pusat
		\item 192.168.2.0/24: jaringan cabang
	\end{itemize}

	\textbf{Referensi}
	\begin{itemize}
		\item Cisco Documentation: \href{https://www.cisco.com/c/en/us/td/docs/ios-xml/ios/sec_conn_vpnips/configuration/xe-16/sec-vpnips-xe-16-book/sec-vpn-site.html}{IPSec Configuration Guide}
		\item RFC 7296 - \href{https://tools.ietf.org/html/rfc7296}{Internet Key Exchange Protocol Version 2 (IKEv2)}
	\end{itemize}

	\item \textbf{Skema Queue Tree}
	
	Total bandwidth: 100 Mbps. Kita bagi menggunakan Queue Tree di MikroTik.

	\textbf{Parent Queue (Interface WAN):}
	\begin{itemize}
		\item queue tree utama menggunakan max-limit=100M
	\end{itemize}
	\begin{lstlisting}
		/queue tree
		add name="total-bandwidth" parent=ether1 max-limit=100M
	\end{lstlisting}

	\textbf{Child Queues:}
	\begin{lstlisting}
		add name="e-learning" parent="total-bandwidth" limit-at=40M max-limit=40M priority=1 packet-mark=e-learning
		add name="guru-staf" parent="total-bandwidth" limit-at=30M max-limit=30M priority=3 packet-mark=guru-staf
		add name="siswa" parent="total-bandwidth" limit-at=20M max-limit=20M priority=5 packet-mark=siswa
		add name="cctv-update" parent="total-bandwidth" limit-at=10M max-limit=10M priority=8 packet-mark=cctv-update
	\end{lstlisting}

	\textbf{Penjelasan Marking (Firewall Mangle):}

	Digunakan untuk menandai lalu lintas sebelum diteruskan ke queue tree.
	\begin{lstlisting}
		/ip firewall mangle
		add chain=forward src-address=192.168.10.0/24 action=mark-packet new-packet-mark=e-learning passthrough=no
		add chain=forward src-address=192.168.20.0/24 action=mark-packet new-packet-mark=guru-staf passthrough=no
		add chain=forward src-address=192.168.30.0/24 action=mark-packet new-packet-mark=siswa passthrough=no
		add chain=forward src-address=192.168.40.0/24 action=mark-packet new-packet-mark=cctv-update passthrough=no
	\end{lstlisting}

	\begin{center}
		\begin{tabular}{ |p{3cm}||p{3cm}||p{3cm}| }
			\hline
			\multicolumn{3}{|c|}{Parameter Keamanan yang Harus Disepakati} \\
			\hline
			SUBNET & PENGGUNA & MARK \\
			\hline
			192.168.10.0/24	 & e-learning	& e-learning \\
			192.168.20.0/24	 & guru \& staf	& guru-staf \\
			192.168.30.0/24	 & siswa 		& siswa \\
			192.168.40.0/24	 & CCTV/update	& cctv-update \\
			\hline
		\end{tabular}
	\end{center}

	\textbf{Prioritas dan Limit Rate}
	\begin{itemize}
		\item Priority (1 = tertinggi, 8 = terendah):
		\begin{itemize}
			\item e-learning (1): akses penting
			\item guru-staf (3): cukup penting
			\item siswa (5): browsing umum
			\item CCTV/update (8): boleh diundur saat sibuk
		\end{itemize}
		\item Limit-at: bandwidth minimum terjamin
		\item Max-limit: batas maksimum \newline
	\end{itemize} 

	\textbf{Referensi}
	\begin{itemize}
	\item MikroTik Wiki - \href{https://wiki.mikrotik.com/wiki/Manual:Queue}{Queue Tree}
	\item MikroTik Documentation - \href{https://wiki.mikrotik.com/wiki/Manual:IP/Firewall/Mangle}{Mangle}
	\item MikroTik Forum: \href{https://forum.mikrotik.com/viewtopic.php?t=133621}{Best Practice Bandwidth Management}		
	\end{itemize}

\end{enumerate}