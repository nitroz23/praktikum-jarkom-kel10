\section{Pendahuluan}
\subsection{Latar Belakang}
Seiring dengan penggunaan internet yang berkembang pesat, 
kebutuhan akan lebih banyak alamat IP telah menjadi masalah 
penting. Skema pengalamatan IPv4, yang menggunakan alamat 32-
bit, menyediakan sekitar 4,3 miliar alamat IP yang unik. 
Meskipun jumlah tersebut tampaknya cukup memadai pada masa-masa 
awal internet, ledakan perangkat yang terhubung - seperti ponsel 
pintar, sistem IoT, dan layanan cloud - telah menyebabkan ruang 
alamat IPv4 yang tersedia hampir habis.

Untuk mengatasi keterbatasan ini, Internet Protocol versi 6 
(IPv6) diperkenalkan oleh Internet Engineering Task Force (IETF). 
IPv6 menggunakan alamat 128-bit, secara signifikan meningkatkan 
jumlah alamat IP yang memungkinkan menjadi sekitar $3,4\times 10^{38}$, 
yang lebih dari cukup untuk mengakomodasi perluasan internet di 
masa mendatang. IPv6 tidak hanya menawarkan ruang alamat yang 
jauh lebih besar, tetapi juga mencakup peningkatan dalam 
efisiensi perutean, fitur keamanan, dan header paket yang 
disederhanakan.
\subsection{Dasar Teori}
IPv6 (Internet Protocol versi 6) adalah versi terbaru dari 
Internet Protocol, yang bertanggung jawab untuk mengidentifikasi 
dan menemukan komputer di jaringan dan merutekan lalu lintas di 
internet. Tidak seperti pendahulunya, IPv4, yang menggunakan 
alamat 32-bit dan mendukung sekitar 4,3 miliar alamat IP unik, 
IPv6 menggunakan alamat 128-bit, yang memungkinkan sekitar 340 
undecillion ($3,4\times 10^{38}$) kombinasi unik. Peningkatan ruang 
alamat yang sangat besar ini diperlukan untuk mengakomodasi 
pertumbuhan pesat perangkat yang terhubung ke internet. Alamat 
IPv6 ditulis dalam heksadesimal dan dipisahkan dengan titik dua, 
seperti 2001:0db8:85a3:0000:0000:8a2e:0370:7334, dan dapat 
dipersingkat menggunakan aturan notasi standar untuk memudahkan.

Selain kapasitas alamat yang lebih besar, IPv6 menawarkan 
beberapa peningkatan dibandingkan IPv4. IPv6 memiliki struktur 
header paket yang disederhanakan yang memungkinkan pemrosesan 
yang lebih efisien oleh router, mendukung mekanisme konfigurasi 
otomatis seperti SLAAC (Stateless Address Autoconfiguration) dan 
DHCPv6, dan termasuk dukungan bawaan untuk IPsec, meningkatkan 
keamanan data. IPv6 juga menghilangkan kebutuhan akan Network 
Address Translation (NAT), sebuah solusi yang biasanya digunakan 
pada IPv4 karena ketersediaan alamat yang terbatas. Dengan 
memungkinkan konektivitas langsung dari ujung ke ujung, IPv6 
memfasilitasi kinerja yang lebih baik untuk aplikasi seperti 
komunikasi waktu nyata, layanan cloud, dan penyebaran Internet 
of Things (IoT). Kemajuan ini menjadikan IPv6 sebagai komponen 
penting dalam mendukung skalabilitas, keandalan, dan keamanan 
infrastruktur internet global di masa depan.
%===========================================================%
\section{Tugas Pendahuluan}
Bagian ini berisi jawaban dari tugas pendahuluan yang telah anda kerjakan, beserta penjelasan dari jawaban tersebut
\begin{enumerate}
	\item IPv6, atau Internet Protocol versi 6, adalah versi 
	terbaru dari Internet Protocol yang dirancang untuk 
	mengatasi keterbatasan IPv4, khususnya kekurangan alamat IP 
	yang tersedia. Sementara IPv4 menggunakan alamat 32-bit dan 
	mendukung sekitar 4,3 miliar alamat IP unik, IPv6 
	menggunakan alamat 128-bit, sehingga memungkinkan jumlah 
	alamat unik yang hampir tak terbatas-sekitar 340 undecillion. 
	Hal ini membuat IPv6 ideal untuk mendukung semakin banyaknya 
	perangkat yang terhubung ke internet. Selain ruang alamat 
	yang diperluas, IPv6 memperkenalkan beberapa peningkatan, 
	seperti header paket yang disederhanakan, dukungan bawaan 
	untuk keamanan melalui IPSec, dan dukungan yang lebih baik 
	untuk konfigurasi otomatis menggunakan SLAAC (Konfigurasi 
	Otomatis Alamat Tanpa Kewarganegaraan). IPv6 juga 
	menghilangkan kebutuhan akan NAT (Network Address 
	Translation), yang biasanya digunakan dalam IPv4 untuk 
	menghemat ruang alamat. Secara keseluruhan, IPv6 dirancang 
	agar lebih terukur, aman, dan efisien, sehingga lebih cocok 
	untuk internet modern.
	\item \begin{enumerate}
		\item Karena kita memiliki 32 bit, kita dapat menggunakan 
		32 bit lagi untuk subnetting untuk mencapai 64 bit.
		\begin{itemize}
			\item Subnet 1: 2001:db8:0000:0000::/64
			\item Subnet 2: 2001:db8:0000:0001::/64
			\item Subnet 3: 2001:db8:0000:0002::/64
			\item Subnet 4: 2001:db8:0000:0003::/64
		\end{itemize}
		\item Berikut alokasi alamat IPv6 untuk masing-masing 
		subnet:
		{\small
		\begin{center}
		\begin{tabular}{ |c|c| } 
			\hline
			Subnet & Alamat Subnet  \\
			\hline
			Subnet A & 2001:db8:0:0::/64\\
			Subnet B & 2001:db8:0:1::/64 \\
			Subnet C & 2001:db8:0:2::/64  \\
			Subnet D & 2001:db8:0:3::/64   \\
			\hline
		\end{tabular}
		\end{center}
		}
	\end{enumerate}
	\item \begin{enumerate}
		\item Alamat IPv6 pada setiap antarmuka router adalah 
		sebagai berikut:
		{\small
		\begin{center}
		\begin{tabular}{ |c|c|c| } 
			\hline
			Interface & Assigned Subnet & IPv6 Address on Router Interface  \\
			\hline
			Ether 1 & Subnet A & 2001:db8:0:0::1/64\\
			Ether 2 & Subnet B & 2001:db8:0:1::1/64 \\
			Ether 3 & Subnet C & 2001:db8:0:2::1/64  \\
			Ether 4 & Subnet D & 2001:db8:0:3::1/64   \\
			\hline
		\end{tabular}
		\end{center}
		}
		\item Untuk konfigurasi antarmuka router, dapat dilakukan 
		dalam program seperti mikrotik. Command CLI yang digunakkan 
		antara lain seperti berikut:
		{\small
		\begin{center}
		\begin{tabular}{ |l| } 
			\hline
			/ipv6 address \\
			add address=2001:db8:0:0::1/64 interface=ether1 \\
			add address=2001:db8:0:1::1/64 interface=ether2 \\
			add address=2001:db8:0:2::1/64 interface=ether3 \\
			add address=2001:db8:0:3::1/64 interface=ether4 \\
			\hline
		\end{tabular}
		\end{center}
		}
	\end{enumerate}
	\item Berikut IP table routing statis agar semua subnet saling 
	berkomunikasi:
	{\small
		\begin{center}
		\begin{tabular}{ |c|c|c|c| } 
			\hline
			Destination Network & Gateway & Interface & Deskripsi  \\
			\hline
			2001:db8:0000:0100::/64	 & 2001:db8:0000:0100::1	 & ether1 & Menuju B melalui ether2	\\
			2001:db8:0000:0200::/64	 & 2001:db8:0000:0200::1	 & ether1 & Menuju C melalui ether3	 \\
			2001:db8:0000:0300::/64	 & 2001:db8:0000:0300::1	 & ether1 & Menuju D melalui ether4	  \\
			2001:db8:0000:0000::/64	 & 2001:db8:0000:0000::1	 & ether2 & Menuju A melalui ether1	   \\
			2001:db8:0000:0200::/64	 & 2001:db8:0000:0200::1	 & ether2 & Menuju C melalui ether3	\\
			2001:db8:0000:0300::/64	 & 2001:db8:0000:0300::1	 & ether2 & Menuju D melalui ether4	 \\
			2001:db8:0000:0000::/64	 & 2001:db8:0000:0000::1	 & ether3 & Menuju A melalui ether1	  \\
			2001:db8:0000:0100::/64	 & 2001:db8:0000:0100::1	 & ether3 & Menuju B melalui ether2	   \\
			2001:db8:0000:0300::/64	 & 2001:db8:0000:0300::1	 & ether3 & Menuju D melalui ether4	\\
			2001:db8:0000:0000::/64	 & 2001:db8:0000:0000::1	 & ether4 & Menuju A melalui ether1	 \\
			2001:db8:0000:0100::/64	 & 2001:db8:0000:0100::1	 & ether4 & Menuju B melalui ether2	  \\
			2001:db8:0000:0200::/64	 & 2001:db8:0000:0200::1	 & ether4 & Menuju C melalui ether3	   \\
			\hline
		\end{tabular}
		\end{center}
		}
	\item Perutean statis dalam jaringan IPv6 mengacu pada 
	konfigurasi manual jalur perutean antar jaringan oleh 
	administrator jaringan. Alih-alih mengandalkan protokol 
	dinamis untuk secara otomatis berbagi informasi perutean, 
	rute statis ditentukan sebelumnya dan tetap, sehingga 
	administrator memiliki kontrol penuh atas bagaimana lalu 
	lintas mengalir di antara subnet atau jaringan. Metode ini 
	sangat berguna di jaringan yang lebih kecil atau kurang 
	kompleks di mana rute jarang berubah, karena memastikan 
	prediktabilitas dan stabilitas. Routing statis juga lebih 
	disukai dalam situasi di mana keamanan dan kontrol sangat 
	penting, seperti di lingkungan server atau ketika menguji 
	perilaku jaringan tertentu. Namun, ini menjadi kurang 
	praktis dalam jaringan yang lebih besar atau lebih dinamis, 
	di mana mempertahankan konfigurasi rute manual dapat 
	memakan waktu dan rentan terhadap kesalahan. Dalam kasus 
	seperti itu, protokol routing dinamis seperti RIPng, 
	OSPFv3, atau BGP lebih cocok, karena dapat secara otomatis 
	beradaptasi dengan perubahan jaringan tanpa intervensi 
	manual.
\end{enumerate}