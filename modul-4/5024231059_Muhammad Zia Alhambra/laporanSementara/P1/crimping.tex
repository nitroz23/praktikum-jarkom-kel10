\section{Pendahuluan}
\subsection{Latar Belakang}
Dalam pengembangan dan pemeliharaan jaringan komputer, terutama 
yang terhubung ke Internet, dua konsep penting memainkan peran 
sentral dalam memastikan efisiensi operasional dan keamanan: 
firewall dan Network Address Translation (NAT). Seiring dengan 
meningkatnya permintaan akan perangkat yang terhubung ke 
Internet, administrator jaringan harus menghadapi dua tantangan 
utama—kekurangan alamat IP dan meningkatnya ancaman siber. NAT 
dan firewall, meskipun memiliki tujuan yang berbeda secara 
fundamental, sering kali diimplementasikan bersama dalam gateway 
jaringan atau router untuk mengatasi masalah ini secara efektif.

Firewall adalah mekanisme keamanan jaringan yang dirancang untuk 
memantau dan mengatur lalu lintas yang masuk dan keluar dari 
jaringan berdasarkan kumpulan aturan yang telah ditentukan. 
Berperan sebagai penghalang pelindung antara jaringan internal 
yang tepercaya dan jaringan eksternal yang tidak tepercaya 
(seperti Internet), firewall memainkan peran kritis dalam 
mencegah akses tidak sah, mendeteksi aktivitas berbahaya, dan 
menegakkan kebijakan keamanan organisasi. Firewall dapat 
diimplementasikan dalam bentuk perangkat keras, perangkat lunak, 
atau kombinasi keduanya. Tergantung pada kompleksitas dan 
desainnya, firewall tersedia dalam beberapa jenis, termasuk 
firewall penyaringan paket, firewall inspeksi status, firewall 
lapisan aplikasi, dan firewall generasi berikutnya (NGFW).

\subsection{Dasar Teori}
Firewall penyaring paket beroperasi dengan memeriksa header 
setiap paket dan membuat keputusan penyaringan berdasarkan 
atribut seperti alamat IP sumber dan tujuan, nomor port, serta 
protokol transportasi yang digunakan. Firewall berstatus 
(stateful firewalls) lebih canggih, karena tidak hanya memeriksa 
header paket tetapi juga memelihara status koneksi aktif, 
memungkinkan mereka membuat keputusan yang lebih sadar konteks. 
Firewall lapisan aplikasi memperdalam pemeriksaan ini dengan 
menganalisis lalu lintas yang terkait dengan aplikasi tertentu, 
seperti HTTP atau FTP, memungkinkan deteksi ancaman yang lebih 
canggih. Solusi paling komprehensif, yang dikenal sebagai Next-
Generation Firewalls, menggabungkan fungsi firewall tradisional 
dengan kemampuan canggih seperti sistem deteksi/pencegahan 
intrusi (IDS/IPS), pemeriksaan paket mendalam, dan intelijen 
ancaman real-time.

Dasar teoretis firewall terletak pada penggunaan logika berbasis 
aturan. Aturan-aturan ini, sering kali disusun sebagai daftar 
kontrol akses (ACL), mendefinisikan kondisi di mana paket 
diizinkan atau ditolak untuk melewati. Firewall menerapkan 
logika Boolean untuk mengevaluasi apakah paket tertentu sesuai 
dengan aturan, sehingga menerapkan postur keamanan yang 
diinginkan. Mereka juga dapat mempertahankan tabel koneksi saat 
ini untuk membedakan antara sesi baru dan yang sudah ada, 
terutama dalam model inspeksi berbasis status.

Meskipun firewall berfokus pada pengendalian lalu lintas untuk tujuan 
keamanan, Network Address Translation (NAT) menangani tantangan 
yang berbeda: ketersediaan alamat IPv4 yang terbatas. NAT 
memungkinkan beberapa perangkat di jaringan lokal pribadi (LAN) 
mengakses internet menggunakan satu alamat IP publik. NAT 
mencapai ini dengan memodifikasi informasi alamat IP dalam 
header paket saat paket berpindah antara jaringan pribadi dan 
publik. Ada beberapa jenis NAT, termasuk static NAT, dynamic NAT, 
dan Port Address Translation (PAT), yang juga dikenal sebagai 
NAT overload.

Static NAT menyediakan pemetaan satu-ke-satu antara alamat IP 
pribadi dan publik, biasanya digunakan ketika perangkat di 
jaringan internal perlu diakses dari luar, seperti server web 
atau email. NAT dinamis mengalokasikan alamat IP publik dari 
kumpulan alamat yang tersedia ke perangkat internal sesuai 
kebutuhan, sementara PAT memungkinkan banyak perangkat berbagi 
satu alamat IP publik dengan membedakan sesi berdasarkan nomor 
port. Bentuk terakhir ini paling sering digunakan di jaringan 
rumah dan bisnis kecil karena efisiensinya dalam menghemat 
alamat IP.

Dari perspektif teoretis, NAT beroperasi dengan memodifikasi 
bidang sumber atau tujuan dalam header IP dan TCP/UDP paket, 
sesuai dengan tabel terjemahan yang dikelola oleh router atau 
perangkat gateway. Meskipun NAT menyediakan lapisan keamanan tidak 
langsung dengan menyembunyikan alamat IP internal, ini bukanlah 
fitur keamanan secara langsung; melainkan alat untuk mengelola 
penomoran IP dan memfasilitasi konektivitas. Oleh karena itu, 
mengandalkan NAT saja tanpa menerapkan kebijakan firewall akan 
mengekspos jaringan pada risiko keamanan yang signifikan.

Kombinasi fungsi NAT dan firewall umum ditemukan di sebagian 
besar router komersial dan perangkat keamanan. Dalam konfigurasi 
semacam itu, NAT mengelola penerjemahan alamat untuk koneksi 
keluar, sementara firewall menentukan lalu lintas mana yang 
diizinkan atau ditolak berdasarkan aturan keamanan. Bersama-sama, 
keduanya memberikan manfaat ganda: NAT memastikan penggunaan 
alamat IP yang efisien dan skalabel, sementara firewall 
menerapkan perbatasan keamanan yang melindungi sistem internal 
dari ancaman eksternal.

Namun, jika router tidak memiliki kemampuan penyaringan firewall 
sama sekali, jaringan menjadi sangat rentan terhadap berbagai 
ancaman. Tanpa penyaringan paket atau kontrol akses, lalu lintas 
berbahaya dari internet dapat mencapai perangkat internal tanpa 
batasan. Hal ini dapat menyebabkan akses tidak sah, kebocoran 
data, atau penyebaran malware dan ransomware di dalam jaringan. 
Selain itu, ketidakhadiran kontrol lalu lintas keluar berarti 
perangkat yang terkompromi di dalam jaringan dapat dengan mudah 
mentransmisikan data sensitif atau berkomunikasi dengan server 
komando dan kontrol di luar organisasi, memperburuk dampak dari 
pelanggaran keamanan apa pun.

Kesimpulannya, baik NAT maupun firewall merupakan komponen yang 
tak tergantikan dalam arsitektur jaringan modern. NAT memainkan 
peran vital dalam memfasilitasi komunikasi jaringan pribadi 
dengan internet publik sambil menghemat ruang alamat IP global. 
Di sisi lain, firewall sangat penting untuk menerapkan kebijakan 
keamanan dan melindungi jaringan dari ancaman internal maupun 
eksternal. Bersama-sama, keduanya membentuk lapisan dasar dari 
desain jaringan yang tangguh.  

%===========================================================%
\section{Tugas Pendahuluan}
\begin{enumerate}
	\item Untuk mengakses server web lokal (IP: 192.168.1.10, 
	port 80) dari jaringan eksternal, Anda perlu mengonfigurasi 
	Port Forwarding pada router Anda, yang merupakan jenis 
	Destination NAT (DNAT).

	\textbf{Langkah konfigurasi:}
	\begin{itemize}
		\item Masuk ke antarmuka admin router Anda.
		\item Navigasi ke bagian Port Forwarding atau NAT.
		\item Buat aturan baru:
		\begin{itemize}
			\item Port Eksternal: 80
			\item Protokol: TCP (karena HTTP menggunakan TCP)
			\item IP Internal: 192.168.1.10
			\item Port Internal: 80
		\end{itemize}
		\item Pastikan alamat IP publik router bersifat statis 
		atau gunakan Dynamic DNS (DDNS) jika tidak.
	\end{itemize}

	\textbf{Referensi:} 
	\begin{itemize}
		\item Kurose, J. F., \& Ross, K. W. (2021). Computer 
		Networking: A Top-Down Approach (8th ed.). Pearson.
		\item Cisco. Port Forwarding and NAT Overview
	\end{itemize}
	\item Firewall harus diimplementasikan terlebih dahulu. 
	Firewall adalah mekanisme keamanan yang mengontrol lalu 
	lintas berdasarkan aturan (izinkan/tolak). NAT, meskipun 
	menawarkan keamanan terbatas melalui penyamaran, pada 
	dasarnya adalah metode untuk penerjemahan alamat, bukan 
	penegakan keamanan. Jika Anda mengimplementasikan NAT tanpa 
	firewall, lalu lintas yang tidak difilter masih dapat 
	mencapai sistem internal melalui port yang diteruskan atau 
	celah keamanan. Firewall melindungi jaringan Anda terlepas 
	dari jenis alamat (publik atau privat), sementara NAT hanya 
	memberikan perlindungan tidak langsung yang minimal. 	

	\textbf{Referensi:} 
	\begin{itemize}
		\item Stallings, W. (2013). Network Security Essentials: 
		Applications and Standards (5th ed.). Pearson.
		\item Pfleeger, C. P., \& Pfleeger, S. L. (2015). Security 
		in Computing (5th ed.). Pearson.
	\end{itemize}
	\item Berikut dampak negatif bila router tidak diberi filter 
	firewall:
	\begin{enumerate}
		\item Peningkatan Kerentanan Terhadap Serangan:
		\begin{itemize}
			\item Tanpa penyaringan paket, setiap koneksi masuk 
			(misalnya, pemindaian port, uji coba malware) dapat 
			mencapai perangkat internal.
			\item Contoh: Serangan eksploit jarak jauh, serangan 
			DoS, login paksa.
		\end{itemize}
		\item Akses Tidak Sah:
		\begin{itemize}
			\item Layanan terbuka (seperti SSH, HTTP, SMB) mungkin 
			dapat diakses, terutama jika NAT/pengalihan port 
			dikonfigurasi.
		\end{itemize}
		\item Bocornya Data:
		\begin{itemize}
			\item Tanpa kontrol atas lalu lintas keluar, 
			informasi sensitif dapat dikirimkan oleh malware 
			atau perangkat yang terkompromi.
		\end{itemize}
		\item Penyebaran Worm Jaringan atau Malware dengan 
		Mudah:
		\begin{itemize}
			\item Tanpa segmentasi atau penegakan aturan, 
			malware dapat bergerak secara lateral antar 
			perangkat.
		\end{itemize}
		\item Tidak Ada Logging atau Peringatan:
		\begin{itemize}
			\item Tanpa firewall, Anda tidak memiliki 
			visibilitas terhadap lalu lintas mencurigakan atau 
			berbahaya.
		\end{itemize}
	\end{enumerate}
	\textbf{Referensi:}
	\begin{itemize}
		\item Northcutt, S. (2005). Inside Network Perimeter 
		Security (2nd ed.). Sams Publishing.
		\item NIST SP 800-41 Rev. 1 - Guidelines on Firewalls 
		and Firewall Policy
	\end{itemize}
\end{enumerate}