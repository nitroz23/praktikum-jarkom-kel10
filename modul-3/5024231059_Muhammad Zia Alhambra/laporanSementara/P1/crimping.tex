\section{Pendahuluan}
\subsection{Latar Belakang}
Jaringan nirkabel merupakan komponen penting dari komunikasi 
digital kontemporer, yang memungkinkan transfer data tanpa 
menggunakan kabel fisik. Asal-usulnya dapat ditelusuri kembali 
ke karya dasar elektromagnetisme selama abad ke-19, khususnya 
eksperimen Heinrich Hertz, yang pertama kali menunjukkan 
keberadaan gelombang radio. Tak lama kemudian, Guglielmo Marconi 
mengembangkan karya ini untuk mengembangkan sistem telegraf 
nirkabel praktis pertama, yang menandai dimulainya komunikasi 
nirkabel. Seiring berjalannya waktu, inovasi awal ini meletakkan 
dasar bagi berbagai teknologi nirkabel, mulai dari siaran radio 
dasar hingga jaringan seluler modern dan sistem Wi-Fi. 
Daya tarik utama komunikasi nirkabel selalu terletak pada 
kemampuannya untuk mengirimkan informasi jarak jauh tanpa 
bergantung pada koneksi kabel, yang mengurangi kebutuhan 
infrastruktur dan meningkatkan fleksibilitas.

Penerapan jaringan nirkabel secara luas meningkat pada akhir 
abad ke-20 dan awal abad ke-21 dengan munculnya teknologi Wi-Fi 
dan pita lebar seluler. Kemajuan ini memungkinkan akses internet 
tanpa hambatan di rumah, kantor, dan ruang publik, yang secara 
mendasar mengubah cara orang berinteraksi dengan informasi 
digital. Seiring dengan makin lazimnya penggunaan perangkat 
seluler seperti telepon pintar, tablet, dan laptop, permintaan 
akan konektivitas yang konstan dan bergerak pun meningkat pesat. 
Jaringan nirkabel memungkinkan hal ini dengan menyediakan solusi 
yang mudah, dapat diskalakan, dan relatif murah untuk mengakses 
sumber daya jaringan. Saat ini, jaringan nirkabel sangat penting 
dalam berbagai sektor, termasuk pendidikan, perawatan kesehatan, 
transportasi, dan bisnis, yang mendorong inovasi dan efisiensi 
baik di lingkungan perkotaan maupun terpencil.
\subsection{Dasar Teori}
Pengoperasian jaringan nirkabel didasarkan pada prinsip-prinsip 
teori gelombang elektromagnetik. Komunikasi nirkabel bekerja 
dengan mengirimkan data melalui frekuensi radio (RF), yang 
merupakan bagian dari spektrum elektromagnetik. Sinyal RF ini 
membawa informasi digital atau analog dari pemancar ke penerima. 
Salah satu proses inti yang terlibat adalah modulasi, yang 
memodifikasi gelombang pembawa untuk mengodekan data yang sedang 
dikirim. Berbagai teknik modulasi—seperti modulasi amplitudo 
(AM), modulasi frekuensi (FM), atau skema modulasi digital yang 
lebih canggih seperti QAM (Quadrature Amplitude Modulation)—digunakan 
untuk meningkatkan efisiensi dan keandalan. Sinyal yang dikirim 
menyebar melalui udara dan dapat dipengaruhi oleh berbagai 
faktor lingkungan seperti rintangan, kondisi atmosfer, dan 
perangkat elektronik lainnya.

Beberapa fenomena memengaruhi kinerja dan perilaku jaringan 
nirkabel, termasuk redaman, interferensi, dan propagasi 
multijalur. Redaman mengacu pada melemahnya kekuatan sinyal saat 
bergerak melintasi jarak atau melalui material seperti dinding 
dan kaca. Interferensi terjadi ketika perangkat elektronik lain 
atau sinyal yang tumpang tindih mengganggu komunikasi, yang 
menyebabkan hilangnya data atau kecepatan berkurang. Perambatan 
multijalur terjadi saat sinyal terpantul dari permukaan dan 
mengambil beberapa jalur untuk mencapai penerima, yang 
menyebabkan penundaan atau distorsi sinyal. Untuk mengatasi 
tantangan ini, jaringan nirkabel mematuhi standar dan protokol 
tertentu seperti IEEE 802.11 (Wi-Fi), yang menentukan cara 
perangkat berkomunikasi, menghindari tabrakan, dan memastikan 
pertukaran data yang aman. Protokol ini menggunakan teknik 
seperti alokasi saluran, koreksi kesalahan, dan enkripsi untuk 
menjaga komunikasi yang efisien dan aman. Kombinasi elemen 
teoritis ini membentuk tulang punggung ilmiah yang memungkinkan 
jaringan nirkabel berkecepatan tinggi yang andal dalam kehidupan 
sehari-hari.
%===========================================================%
\section{Tugas Pendahuluan}
\begin{enumerate}
	\item Jaringan berkabel umumnya menawarkan kecepatan yang 
	lebih cepat, keandalan yang lebih tinggi, dan keamanan yang 
	lebih baik karena tidak terlalu rentan terhadap gangguan dan 
	penyadapan. Jaringan ini sangat ideal untuk lingkungan di 
	mana kinerja dan stabilitas sangat penting, seperti kantor 
	atau pusat data. Namun, jaringan ini tidak memiliki 
	fleksibilitas dan membutuhkan pemasangan kabel fisik, yang 
	mungkin merepotkan atau mahal untuk dipasang. Sebaliknya, 
	jaringan nirkabel memberikan kenyamanan dan mobilitas yang 
	lebih besar, memungkinkan pengguna untuk menghubungkan 
	perangkat tanpa kabel fisik. Hal ini membuatnya ideal untuk 
	rumah, area publik, atau situasi di mana pemasangan kabel 
	tidak praktis. Namun, koneksi nirkabel bisa jadi kurang 
	stabil dan lebih rentan terhadap gangguan dan risiko 
	keamanan.
	\item Modem, router, dan titik akses masing-masing memiliki 
	peran yang berbeda dalam suatu jaringan. 
	\begin{enumerate}
		\item Modem adalah perangkat yang menghubungkan 
		jaringan lokal ke internet dengan menerjemahkan sinyal 
		antara Penyedia Layanan Internet (ISP) dan jaringan rumah 
		atau kantor.
		\item Router berada di antara modem dan perangkat, 
		mendistribusikan koneksi internet ke beberapa perangkat 
		baik melalui kabel Ethernet maupun nirkabel. Router juga 
		mengelola lalu lintas jaringan dan dapat menyertakan fitur 
		seperti firewall atau kontrol orangtua.
		\item Titik akses memperluas jangkauan nirkabel suatu 
		jaringan dengan menghubungkan ke router melalui kabel 
		Ethernet dan menciptakan sinyal Wi-Fi baru. Router 
		umumnya digunakan untuk meningkatkan akses nirkabel di 
		gedung-gedung besar atau bertingkat.
	\end{enumerate}  
	\item Jika diminta untuk menghubungkan dua ruangan di gedung 
	yang berbeda tanpa menggunakan kabel, pilihan terbaik adalah 
	jembatan nirkabel atau sistem nirkabel titik-ke-titik (PtP). 
	Pengaturan ini menggunakan antena terarah untuk membangun 
	hubungan nirkabel yang terfokus antara dua lokasi, 
	seringkali dalam jarak yang jauh. Perangkat ini secara 
	khusus dirancang untuk penggunaan di luar ruangan dan dapat 
	memberikan koneksi yang stabil dan berkecepatan tinggi yang 
	sebanding dengan kabel fisik. Perangkat ini jauh lebih 
	efektif daripada Wi-Fi biasa untuk komunikasi antar gedung 
	karena meminimalkan gangguan dan memfokuskan sinyal dalam 
	garis lurus antara dua titik akses. Hal ini membuatnya ideal 
	untuk memperluas jaringan antara bangunan yang terpisah 
	tanpa perlu menggali parit atau memasang kabel.
\end{enumerate}